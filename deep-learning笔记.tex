\documentclass[UTF8]{ctexart}
\usepackage{geometry}
\geometry{a4paper,scale=0.8}
\sectionfont{\bfseries\Large\raggedright}

\title{deep-learning笔记}
\author{徐世桐}
\date{}
\begin{document}
\maketitle

\section{Chapter 6}

\noindent \textbf{SVM支持向量机}

  仍通过$w^Tx+b$得到输出,输出仅表示identity,正值说明有identity,负值说明没有

  依据:一个平面的公式为$\beta_0+\beta_1x_1+\beta_2x_2=0$,则当计算$w^Tx+b$得到值后,>0则为平面上方的数据点,<0为下方数据点\\
\textbf{kernel trick}

  kernel method将数据集表示成相近的两个数据点一组的集合$(x_i, x_j)$,kernel method将一对数据变为单一数据点$x=k(x_i, x_j)=\langle \phi (x_i), \phi (x_j)\rangle $

  kernel method使用$\phi $转换数据的纬度,而点乘 化简后无需先计算$\phi (x_i), \phi (x_j)$即可得到新数据点$x$\\
\textbf{manifold hypothesis:}

  当训练数据集合包含大量无规律的数据,则将其中大部分视为无效数据,并只关心落在一个manifold上的数据。

  例:生成图像 文字 声音时数据大多很集中,当像素文字随机分布时生成图像大多无意义\\
\textbf{deep feedforward network/feedforward neural network/multilayer perceptrons MLP:}

  找到$\theta$使得$f(x; \theta )$ 最接近数据y值。$f^*$为最理想的f,即$f^*(x) = y$。$\theta $可为多个参数,如$f(x; w, b) = x^Tw+b$

  $f^*(x) = f^{(3)}(f^{(2)}(f^{(1)}(x)))$,$f^{(1)}$为network第一层。每一$f^{(i)}(x) = \phi (x; \theta )^Tw$

  \textbf{神经网络}

  \quad 1. 结构:

  \quad \quad 输入层没有weight,第一hidden layer得到所有输入层的值。

  \quad \quad hidden layer和输出层所有输出都为0/1,非连续的值

  \quad 2. 一层hidden layer计算方法:$ f^{(i)}(x; W, c) = \sigma (W^Tx + c)$

  \quad \quad x 为前一层的输出向量,输入层x即为参数向量。 

  \quad \quad c 为此层常数向量

  \quad \quad z = $W^Tx$为一层hidden layer对输入取得的中间值向量,a = $\sigma (z + c)$为对z + c每一元素取$\sigma $的结果向量,a即此层的输出。

  \quad \quad W 为此层参数矩阵,行数 = 前层节点数,列数 = 当前层节点数

  \quad \quad X 为多个参数点的训练集中前一层的输出矩阵,行数为数据点个数,列数为前一层节点数

  \quad \quad $XW$ 当W对参数集矩阵操作时,每行向量$z_{i}^T$此时为一层hidden layer各节点对第i参数点的中间值向量。对每行+$c^T$并分别取$\sigma $得到输出矩阵,$a_{ij}$为当使用第i个参数点时此层第j节点的输出

\end{document}
